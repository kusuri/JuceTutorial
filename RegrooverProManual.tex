\documentclass[12pt]{article}
% Page setup
\usepackage[top=3cm, bottom=3cm, left=3cm, right=3cm]{geometry}
% Use the Helvetica font
\usepackage{helvet}
\renewcommand{\familydefault}{\sfdefault}
% For images
\usepackage{graphicx}
\usepackage{wrapfig}
% For the cover page
\usepackage{incgraph}
% For fancy keyboard modifiers
\usepackage{menukeys}
% URL handling
\usepackage{url}
% For colored background boxes
\usepackage{xcolor}
\usepackage{mdframed}
% Shaded box styles
\mdfdefinestyle{warning}{
	hidealllines=true,
	backgroundcolor=orange!20,
}

\mdfdefinestyle{info}{
	hidealllines=true,
	backgroundcolor=blue!20,
}

\mdfdefinestyle{tip}{
	hidealllines=true,
	backgroundcolor=green!20,
}

% Command to include images inline with texts
\newcommand{\textimage}[3]{
\raisebox{#1}{\includegraphics[scale=#2]{#3}}
}

\begin{document}
% Cover page
\incgraph[documentpaper] [width=\paperwidth]{./images/cover_essential.png}

% Disclaimer pager
\newpage
The information in this document is subject to change without notice and does not represent a commitment on the part of Accusonus Inc. The software described by this document is subject to a License Agreement and may not be copied to other media. No part of this publication may be copied, reproduced or otherwise transmitted or recorded, for any purpose, without prior written permission by Accusonus Inc, hereinafter referred to as Accusonus. All product and company names are trademarks or registered trademarks of their respective owners.\\
%\\
%Manual version: v1.2 \\
%\\
\\
Accusonus Inc.\\
Lexington, Massachusetts\\
USA\\ 
\newpage

\section*{Welcome to Regroover!}
Regroover is an Artificial-Intelligence beat machine that extends traditional music sampling. Regroover is based on accusonus' patent-pending ``advanced audio analysis'' ($\mathrm{a^3}$) technology which also powers drumatom!

Regroover's A.I. engine analyzes an audio clip and extracts the fundamental audio elements in separate tracks called \textit{layers}. Regroover allows you to reach deep within your audio clips and extract previously unreachable sound elements. For example, if your clip is a drum loop, Regroover will extract layers containing the kick, the snare and the cymbals or any combination of those that is meaningful to the A.I. engine. If you use Regroover with an ambient soundscape loop, the resulting layers will be more abstract (and perharps more interesting).

After Regroover has extracted the layers from your audio clip, it's time for you to start using them to create fresh and exciting music! You can use the layers to go beyond traditional music sampling and kickstart your inspiration. With Regroover your can:
\begin{itemize}
\item Extract layers from different clips and create new, unique audio clips and loops
\item Remix and rearrange the layers to create variations of your audio clip
\item Extract isolated parts from layers and create your own signature sounds and kits
\item Experiment with the settings of the analysis engine to extract inspiring sound elements
\end{itemize}

Regroover is a unique beat machine and instrument that we hope will provide you with new sounds, capabilities and workflows and allow you to create more music, faster and in a very stimulating way! %Please, continue reading the manual to familiarize yourself with this novel product and its features.

\section*{Installation \& activation}
You can find the latest Regroover installer in your accusonus account page in the ``Downloads'' section
(\url{http://accusonus.com/dashboard}). For more information about installation and activation see the Activation Manual, which you can find along with the installer.

\subsubsection*{Requirements}
Your computer must have:
\begin{itemize}
\item an Intel (or Intel compatible) CPU 
\item a minimum of 2GB RAM memory
\item at least 250MB free space on your hard disk
\item a minimum a screen resolution of 1280x800 pixels
\end{itemize}
\subsubsection*{Supported formats}
Regroover is available as a \textbf{virtual instrument} in the following formats:
\begin{itemize}
\item AU (32/64-bit), VST (32/64-bit) and AAX (64-bit) for Mac OS X 10.9 or higher
\item VST (32/64-bit) and AAX (64-bit) for Windows 7 or higher
\end{itemize}
We have tested Regroover in the following DAWs:
\begin{itemize}
\item Ableton Live 9.5
\item Logic Pro 10.2 
\item Steinberg Cubase 7.5 (64-bit only)
\item Avid Pro Tools 11.3 and 12.5
\item Cockos Reaper 5.22
\item Presonus Studio One 3.0
\end{itemize}
Regroover \textbf{is not supported} in the following:
\begin{itemize}
\item ImageLine FL Studio
\item Bitwig Studio
\item Steinberg Cubase 32-bit
\end{itemize}
\begin{mdframed}[style = warning]
There are just too many combinations of DAWs and plug-in formats out there, so there's always something we might have not tested. If you don't see your DAW above, you can always download the \textit{free, fully-functional 14-day trial} and see if Regroover is supported for yourself. If you encounter any problems, let us know at \url{support@accusonus.com}
\end{mdframed}

\section*{Experience Regroover in two minutes!}
You can get started with Regroover in just five steps:
\begin{enumerate}
\item \textbf{Import an audio clip} by clicking \textimage{-7pt}{0.8}{./images/select_clip2.png} (or by drag 'n drop). Regroover will automatically split your clip in layers. By default the number of layers is four. 
\begin{mdframed}[style = info]
The layers always add up to the original clip. 
\end{mdframed}
\item \textbf{Listen to the result} by clicking \textimage{-7pt}{0.8}{./images/preview_button.png}. Use the on \textimage{-5pt}{1}{./images/layer_on.png} and solo \textimage{-5pt}{1}{./images/layer_solo.png} buttons to listen to any layer combination.
%\item \textbf{Listen to the result} by clicking ``PREVIEW'' \includegraphics[scale=0.6]{./images/preview_button.png}. Use the on \includegraphics[scale=1]{./images/layer_on.png} and solo \includegraphics[scale=1]{./images/layer_solo.png} buttons to listen to any layer combination.
\item \textbf{Experiment with the number of layers} using the ``Layers'' slider to change the analysis result.
\begin{mdframed}[style = info]
 Remember to hit ``SPLIT'' for the change to take effect.
\end{mdframed}
\item \textbf{Play the layers using MIDI} and set the start/end \includegraphics[scale=1]{./images/left_marker.png}, \includegraphics[scale=1]{./images/right_marker.png} markers for each layer to start regrooving your clip.
\item \textbf{Route each layer to a DAW channel} using the dropdown menu \textimage{-3pt}{0.8}{./images/layer_out.png}, apply your favorite effects and start remixing your clip.
\end{enumerate}

\section*{Import and playback of audio clips}
\subsection*{Importing an audio clip}
To start using Regroover you need to import an audio clip. Regroover supports .wav, .aif and .aiff files of standard sample rates, with a maximum duration of 30 seconds. You can import an audio clip using the following options:
\begin{itemize}
\item Click on the \textimage{-7pt}{0.8}{./images/select_clip2.png} button (available only in \textit{empty} Regroover instance).
\item Drag \& drop a clip directly onto Regroover.
\item Select ``New Project'' from the \textimage{-7pt}{0.8}{./images/project_button.png} drop down menu.
\end{itemize}
%\begin{center}
%\includegraphics[scale=0.625]{./images/select_clip.png} \includegraphics[scale=1]{./images/project_menu.png}
%\end{center}
Once a clip is imported, Regroover will automatically split it into layers.  The \textit{clip info box} will display the clip’s file name.
%\begin{center}
%\includegraphics[scale=0.625]{./images/select_clip.png} \includegraphics[scale=1]{./images/project_menu.png}
%\end{center}
\begin{mdframed}[style = warning]
Before importing a clip, all playback must be disabled. 
\end{mdframed}
\begin{mdframed}[style = warning]
Each time you import a new clip, Regroover will clear all previous data and work. See ``Saving your work'' section for more information on saving Regroover's state and data.
\end{mdframed}

\subsubsection*{Resampling}
When you import an audio clip, it is automatically resampled to match the DAW project's sample rate, \textit{before} any processing takes place. 
\begin{mdframed}[style = warning]
If the DAW project sample rate changes, the clip will be resampled again and a new analysis will take place. Regroover treats this as a \textit{new} audio clip and all your work will be cleared.
\end{mdframed}

\subsubsection*{Sync}
 \begin{wrapfigure}{r}{0.3\textwidth}
\centering
\includegraphics[scale=1]{./images/sync_area.png}
\end{wrapfigure} Regroover can sync the imported audio clip to your DAW session. When you import a clip, Regroover will  detect the clip's BPM. The estimated value will appear in the \textit{BPM text editor}. By default, Regroover is in ``Sync'' mode which means it will automatically warp the audio clip to match the BPM of your DAW session. You can edit this value, if you are not happy with the estimation. 
\begin{mdframed}[style = warning]
While the ``Sync'' mode is enabled any change in the BPM estimation will warp the audio clip accordingly. Regroover treats this as a \textit{new} audio clip and all your work will be cleared.
\end{mdframed}
If you change the session BPM value, ``Sync'' mode will be automatically disabled. The audio clip will remain warped to the last value of the session BPM.

\subsection*{Audio playback}
\subsubsection*{Preview mode}
You can enable Regroover's preview mode by hitting the \textimage{-7pt}{0.8}{./images/preview_button.png} button. While in this mode, Regroover will loop all layer audio through the layer mixer. Use this mode to monitor a specific layer or combination of layers.
\begin{mdframed}[style = info]
Activating DAW playback will disable preview mode.
\end{mdframed}

\subsubsection*{Audition mode}
The audition mode is helpful when you want to listen to a specific part of a layer. Use the \keys{\shift + \cmd} (Mac) or \keys{\shift + \ctrl} (Win) modifiers and click on a layer. Regroover will playback the audio from the point you clicked to the end of the layer. If you release the keyboard modifiers before the end of the layer, playback will stop . You can click on any part of any layer and playback will begin as long as you hold the modifiers.

\subsubsection*{Layer mixer controls}
Each layer has the following basic mixing controls:
\begin{itemize}
\item \textbf{ON:} \textimage{-5pt}{1}{./images/layer_on.png} Turn a layer's audio track on or off.
\item \textbf{Solo:} \textimage{-5pt}{1}{./images/layer_solo.png} Solo a layer's audio track. By default the Solo is exclusive. You can click while holding the \keys{\cmd} (Mac) or \keys{\ctrl} (Win) modifier to Solo multiple layer tracks.
\begin{mdframed}[style = info]
Solo is post fader, post pan and post mono to stereo.
\end{mdframed}
\item \textbf{Sum to mono}:  \textimage{-5pt}{1}{./images/layer_s2m.png} Sum the stereo data of a layer to a mono output.
\item \textbf{Output routing}:  \textimage{-5pt}{1}{./images/layer_out.png} Regroover supports up to 16 DAW stereo output tracks. You can route each layer track to any of those tracks using the dropdown menu.
\begin{mdframed}[style = info]
See the appendix for information on multiple output instruments in various DAWs.
\end{mdframed}
\item \textbf{Gain and pan}: The gain slider controls the layer track's volume. The pan knob controls the stereo position.
\begin{mdframed}[style = info]
The pan knob operates as a simple balance knob for stereo tracks.
\end{mdframed}
\end{itemize}

Regroover comes with a handy EQ that allows you to quickly shape the tone of each layer. It features three selves (low, mid, high) with adjustable frequency and gain.

\section*{Split your clips in meaningful layers}
\begin{wrapfigure}{l}{0.45\textwidth}
\centering
\includegraphics[scale=1]{./images/analysis_area.png}
\end{wrapfigure} \textit{Split} is the operation during which Regroover analyzes the audio clip you have imported and extracts \textit{layers}. The first split takes place automatically when you import a clip. You can change the resulting layers by changing the analysis parameters, using the annotation tool and performing more splits. To perform a split just hit the ``SPLIT'' button.

\subsubsection*{Analysis parameters}
You can change the result of a split by using the two \textit{analysis} parameters:
\begin{itemize}
\item The ``Layers'' slider allows you to select the number of layers Regroover will use to unmix the audio clip. 
This corresponds roughly to the number of audio elements you wish to extract from the clip. Experiment with different results by changing the value of this slider.
\begin{mdframed}[style = tip]
Select small values for simple audio clips and higher values for more complex clips with many audio elements.
\end{mdframed}
\begin{mdframed}[style = warning]
For this change to take effect you need to click ``SPLIT'' to perform a new analysis of your audio clip.
\end{mdframed}
\item You can use the ``Activity'' slider to indicate how busy your audio clip is. For most cases you can leave it at the default setting, between ``Low'' and ``High''.
\begin{mdframed}[style = tip]
If you have an audio clip with slowly changing material set the slider value to ``Low''. If you have a busy audio clip with many sources set the slider value to ``High''. 
\end{mdframed}
\begin{mdframed}[style = warning]
For this change to take effect you need to click ``SPLIT'' to perform a new analysis of your audio clip.
\end{mdframed}
\end{itemize}

\subsubsection*{Annotation tool}
The \textit{annotation} tool is useful if you want to edit and reshape the layers produced by Regroover. Clicking on the ``eraser'' icon \textimage{-8pt}{1}{./images/eraser_button.png} activates the annotation tool. You can double click on a layer to create an \textit{annotation region}. This indicates that Regroover must \textbf{remove} all audio from that region and assign it to another layer. Note that this is not a simple ``copy/paste'' operation and will change the audio of all layers. A new split must be performed during which Regroover's A.I. engine will take all annotated regions into account and produce new layers. 
\begin{center}
\includegraphics[width=\columnwidth]{./images/annotation_region.png}
\end{center}
You can resize annotation regions by dragging either end. You can move them around by dragging them along the layer. You can use the audition mode (\keys{\shift + \cmd} + click for Mac \keys{\shift + \ctrl} + click for Win inside an annotation region) to playback the region and make sure your selection is correct. You can delete a region by clicking anywhere inside the region while holding the \keys{\shift} modifier. 
\begin{mdframed}[style = warning]
If you turn off the annotation tool, the annotations \textit{are not deleted}, but Regroover will ignore them in the next split.
\end{mdframed}

\subsubsection*{Locking a layer}
If you are satisfied with the sound elements of a layer and you want to keep them untouched you can \textit{lock} the layer by clicking on the ``lock'' icon  \textimage{-6pt}{1}{./images/layer_lock.png} . While a layer is locked, it cannot be annotated and any subsequent splits will not affect the audio data.
\begin{mdframed}[style = tip]
Using locked layers and annotations can help you direct Regroover's output and shape the layers to your liking.
\end{mdframed}
\section*{Regroove your clips}
\subsection*{MIDI and layer markers}
You can playback each layer using MIDI notes. Regroover receives MIDI note messages from the DAW or from an external MIDI controller. Each layer ``listens'' for the corresponding note shown on the layer track (notes C3 through F3 for layers 1 to 6 respectively). 
\begin{mdframed}[style = warning]
The corresponding notes fore Cockos Reaper are C4 through F4.
\end{mdframed}
Regroover has three MIDI modes which you can configure from the ``SETTINGS'' window:
\begin{itemize}
\item \textbf{Trigger mode} triggers layer playback from the beginning to the end and stops after playback reaches the end.
%Example: Pressing and releasing C3 will start the playback of layer 1. While layer 1 is playing press and release C3 to restart layer 1 playback from the top.
\item \textbf{Toggle mode} toggles playback of layers. The first note on MIDI event will begin playback from the beginning and loop until another note on event is received.
%Example: Pressing and releasing C3 will start the playback of Layer 1. Layer 1 will loop until C3 is pressed and released again.
\item \textbf{Hold mode} loops over the layer audio for as long as you hold down the corresponding note. %Example: Layer 1 will only be playback and loop for as long as you hold down the C3 note.
\end{itemize}
\begin{mdframed}[style = info]
Regroover's default MIDI mode is \textbf{hold}.
\end{mdframed}

\begin{mdframed}[style = warning]
MIDI playback (for any mode) is affected by layer mixer controls and effects settings.
\end{mdframed}

The \textit{markers} are very useful when you want to control the beginning and end point of MIDI playback for each layer. You can drag the left \includegraphics[scale=1]{./images/left_marker.png} and right \includegraphics[scale=1]{./images/right_marker.png} markers to select the region you want to playback using the MIDI notes. The markers for each layer are independent, but you can hold the \keys{\shift} modifier while dragging to align all corresponding markers from all layers.
\begin{mdframed}[style = warning]
Preview mode playback \textbf{ignores} marker positions.
\end{mdframed}

\section*{Saving your work}
\subsection*{Saving a DAW project}
DAW projects that contain Regroover instances contain all the required data so you can continue your work, apart from the imported audio clip. 
\begin{mdframed}[style = warning]
If the location of the audio clip is changed the DAW project won't load.
\end{mdframed}

\subsection*{Regroover projects}
A Regroover project is a self-contained file with the \texttt{.regroover} extension. You can save all the required information and data from a Regroover session in such a project. The main advantage of using Regroover projects is that you can move them around from one computer to another, from one DAW to another and share them with anyone that has a copy of Regroover.
%All the info of a Regroover session, from layers and annotations, to the effects’ settings and samples, consist a Regroover Project. Projects are saved as .regroover files so you can load them anytime in any project or DAW!
\begin{center}
\includegraphics[scale=1]{./images/project_menu2.png}
\end{center}
Click the ``PROJECT'' button in the top bar to access the project menu. Select ``Save project'' or ``Save project as...'' to save your Regroover project. Once a project is saved the project's name will appear inside the button. For your convenience, you can select a default directory where all Regroover projects will be saved in the settings window. Select ``Open project...'' to load a Regroover project. %If you select ``New project'' you can import a new audio clip. Note that this will reset all previous data and work.

\subsection*{Exporting layers}
Click on the \textimage{-7pt}{0.8}{./images/export_button.png} button in the top bar to export the layers as audio files (.wav format) to a folder of your choice. The sample rate of the exported layers will be the same as your DAW session’s sample rate. 

\newpage
\section*{Appendix}
Here you can find information on how to enable multiple output instruments in various DAWs.

\subsection*{Ableton Live}
\begin{enumerate}
\item Insert Regroover Essential on a MIDI track.
\item Create one to five new Audio Tracks depending on how many Layers will be created. The Regroover MIDI track carries the audio of Output 1, so if you want to route all Layers separately, the minimum number of tracks you will need to create is one less than the number of your Layers. 
\item Specify which Regroover Layer is routed to which Regroover Output from the Output Routing drop down menu of each Layer. In this example Layer 2 is routed to Regroover Output 10. You can have the same output for more than one Layer, for example Output 5 could carry the audio of both Layer 1 and Layer 3.
\item In Ableton click on the Audio From drop down menu in one of the created Audio Tracks and select Regroover Essential. On the next menu, select the Regroover Output you wish to receive on this track. Remember that the Regroover MIDI track carries the audio of Output 1 and that is why the routing options of the drop down menu start with \# 2-Regroover Essential.
\item Set the Monitor Mode of the Audio Track to “IN”.
\end{enumerate}

\subsection*{Pro Tools}
\begin{enumerate}
\item Insert Regroover Essential on an Instrument Track.
\item Create one to five new Aux Tracks depending on how many Layers will be created. The Regroover MIDI track carries the audio of Output 1, so if you want to route all Layers separately, the minimum number of tracks you will need to create is one less than the number of your Layers. 
\item Specify which Regroover Layer is routed to which Regroover Output from the Output Routing drop down menu of each Layer. In this example Layer 2 is routed to Regroover Output 10. You can have the same output for more than one Layer, for example Output 5 could carry the audio of both Layer 1 and Layer 3.
\item In Pro Tools I/O settings of the created Aux Track, click on the Input drop down menu and select plug-in. Then choose Regroover Essential and the Output you want to receive on this Aux Track. Remember that the Regroover Instrument Track carries the audio of Output 1 and that is why the Input options of the drop down menu start with Output 2.
\end{enumerate}

\subsection*{Steinberg Cubase}
\begin{enumerate}
\item To create a Regroover Essential instance go to Project$\rightarrow$ Add Track$\rightarrow$ Instrument and select Regroover Essential from the list of instruments.
\item In Cubase, if you load a VST into the rack, only the first output channels is enable by default. To add further channels or activate all of Regroover’s Outputs, after Regroover instantiates go to Devices$\rightarrow$ VST Instruments and click on the small arrow icon ( )  next to Regroover. Then, activate the channels you need or select All Outputs from the pop-up menu.
\item Specify which Regroover Layer is routed to which Regroover Output from the Output Routing drop down menu of each Layer. In this example Layer 2 is routed to Regroover Output 10. You can have the same output for more than one Layer, for example Output 5 could carry the audio of both Layer 1 and Layer 3.
\item Click on the arrow on the bottom left of the Instrument track to reveal all Regroover outputs.
\end{enumerate}

\subsection*{Logic X Pro}
\begin{enumerate}
\item To create a Regroover Essential instance go to Track$\rightarrow$ New Software Instrument Track$\rightarrow$ Instrument and insert the multi-output version of Regroover Essential from the list of AU instruments under the accusonus folder.
\item Click View $\rightarrow$ Show Mixer and select the Regroover track. Click on the (+) button to create Aux Tracks. Create as many AUx tracks needed depending on how many Layers you have. The Regroover Instrument track carries the audio of Output 1, so if you want to route all Layers separately, the minimum number of tracks you will need to create is one less than the number of your Layers. 
\item Specify which Regroover Layer is routed to which Regroover Output from the Output Routing drop down menu of each Layer. In this example Layer 2 is routed to Regroover Output 10. You can have the same output for more than one Layer, for example Output 5 could carry the audio of both Layer 1 and Layer 3.
\item Logic will automatically route Output 2 to Aux 1, Output 3 to Aux 2 etc, but you can always reconfigure which Regroover Output will be received in each Aux Track from the input menu of the mixer. Regroover 3-4 corresponds to stereo Output 2, Regroover 5-6 corresponds to stereo Output 3 and so on.
\end{enumerate}

\subsection*{Cockos Reaper}
\begin{enumerate}
\item Select Track $\rightarrow$ Insert virtual Instrument on new track and choose AUi Regroover Essential or VSTi Regroover Essential from the instruments list.
\item A pop-up window will appear asking you if you want to add multiple tracks for the instance. Select Yes, otherwise a Stereo instance of Regroover Essential will be created.
\item Specify which Regroover Layer is routed to which Regroover Output from the Output Routing drop down menu of each Layer. In this example Layer 2 is routed to Regroover Output 10. You can have the same output for more than one Layer, for example Output 5 could carry the audio of both Layer 1 and Layer 3.
\end{enumerate}

\end{document}